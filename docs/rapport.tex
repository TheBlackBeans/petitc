\documentclass{scrartcl}
\usepackage{cmap}
\usepackage[T1]{fontenc}
\usepackage{lmodern}
\usepackage[utf8]{inputenc}
\usepackage[french]{babel}
\usepackage[colorlinks=true]{hyperref}
\usepackage[outputdir=out, cache=false]{minted}
\usepackage{xcolor}

\definecolor{LightGray}{gray}{0.9}

\title{Compilateur de PetitC}
\author{%
  Jean \textsc{Caspar},
  Adrien \textsc{Mathieu}
}
\date{}

\begin{document}
\maketitle{}

\section*{Prérequis et dépendances}
Ce projet est écrit en Rust, et utilise cargo pour le compiler.\par
Par ailleurs, nous utilisons \href{https://github.com/jthulhu/beans}{Beans}
pour l'analyse lexicale et syntaxique. Beans utilise deux fichiers pour définir
une grammaire:
\begin{itemize}
  \item Un fichier de grammaire de l'analyse lexicale, \verb|gmrs/petitc.lx|.
  \item Un fichier de grammaire de l'analyse syntaxique, \verb|gmrs/petitc.gr|.
\end{itemize}
Pour les utiliser, il faut les compiler avec \verb|beans compile lexer gmrs/petitc.lx|
et \verb|beans compile parser --lexer gmrs/petitc.clx gmrs/petitc.gr|. À noter
que, pour ce faire, il faut disposer de Beans installé au préalable. Pour éviter
que le correcteur ait à installer Beans, ces blobs sont fournis, même s'ils
ne correspondent pas à du code source. Ils ne seront pas supprimés par
\verb|make clean|.\par
Si le correcteur est intéressé, et s'il utilise Emacs (ce qu'il est susceptible
de faire), un plugin Emacs (\href{https://github.com/jthulhu/emacs-beans}{beans.el})
est mis à disposition. Il fournit les major mode correspondants aux fichiers
de grammaire lexicale et de grammaire syntaxique. Ce plugin
\href{https://melpa.org/#/beans}{est disponible sur Melpa}.

\section*{Analyse lexicale et syntaxique}
\subsection*{Analyse lexicale}
L'analyse lexicale repose sur l'abstraction en machines virtuelles plutôt que sur
l'abstraction en automates finis. Cela correspond à la déterminisation à la volée
des automates, ce qui permet, sans trop se fatiguer ultérieurement, de pouvoir
autoriser la capture de groupes dans des expressions régulières, et même du
lookahead, tout en conservant une complexité linéaire.\par
L'analyse lexicale est basée sur le contexte fournit par le parseur. Cela
signifie que les lexèmes qui ne servent pas au parseur à un moment donné ne
peuvent pas être reconnus. En particulier, le programme suivant serait compilé
sans problème, malgré le fait que \verb|int| et \verb|return| sont des mots
réservés.
\begin{minted}[
  frame=lines,
  framesep=2mm,
  bgcolor=LightGray,
  linenos,
  ]{c}
int main() {
  int int = 'a';
  int return = int + 5;
  putchar(return);
  return int / return;
}
\end{minted}

\subsection*{Analyse syntaxique}
L'analyse syntaxique utilise l'algorithme d'Earley. En particulier, toutes les
opérations de scan sont retardées jusqu'à la fin du traitement d'un ensemble,
afin de pouvoir fournir au lexer l'ensemble des tokens qui pourraient être
reconnus.\par
L'analyse syntaxique commence par reconnaître l'entrée, ce qui donne une forêt
de toutes les dérivations possibles. Ensuite, grâce aux règles de priorité et
d'associativité déclarées dans la grammaire, Beans construit explicitement
un des arbre de dérivation (en appliquant les actions sémantiques au passage).\par
Les expressions de la forme \verb|e[i]| sont du sucre syntaxique pour \verb|*(e+i)|,
ce qui signifie que ce sont des actions sémantiques de Beans qui font la
transformation.

\section*{Typage}
Le typage annote l'AST d'informations de typage. Si jamais il est impossible
de typer proprement un n\oe{}ud, il obtient le type \verb|{erreur}|. En général,
si au moins un enfant d'un n\oe{}ud n'est pas bien typé (c'est-à-dire, a le type
\verb|{erreur}|), ce n\oe{}ud aussi a le type erreur. La seule exception à cette règle
est le cas où le type du n\oe{}ud en question ne dépend pas du type de ses enfants.
C'est le cas, par exemple, de la négation logique ($!e$), dont le type est
toujours \verb|int|, indépendemment du type de $e$, ou d'un appel à une fonction,
dont le type de retour ne dépend pas du type des arguments.\par
Une erreur est levée si un n\oe{}ud ne peut pas être bien typé mais tous ses enfants
le sont. À terme, l'AST obtenu en fin de typage sera annoté au minimum, les annotations
utiles à la compilation seront ajoutées dans une autre passe. Pendant cette étape,
on supprime aussi les for avec déclaration de variables et les déclarations de variables
avec assignation. Pendant le typage, on rend unique les noms de toutes
les fonctions en les remplaçant par \verb|fun_NomdelafonctionLigne.Colonne| avec Ligne et Colonne les
emplacements du début de sa déclaration dans le fichier, sauf pour la fonction \verb|main|.

Précédemment, les expressions \verb|++i| et \verb|--i|
étaient remplacées par \verb|i = i + 1| et \verb|i = i - 1|, et les expressions \verb|i++| et \verb|i--| étaient remplacées
par les expressions \verb|(i = i + 1) - 1| et \verb|(i = i - 1) + 1|.\par
En fait, cette dernière étape est fausse. En effet, dans la nouvelle expression,
\(i\) est évaluée deux fois, et non une; si \(i\) a un effet de bord, cela pose
problème. Le programme suivant affiche \verb|a| quand compilé par gcc, et
\verb|b| quand compilé par l'ancienne version de notre compilateur. Ce problème a été résolu
en conservant les expressions \verb|++i|, \verb|i++|, \verb|--i| et \verb|i--| jusqu'à l'étape de production de code.
\begin{minted}[
  frame=lines,
  framesep=2mm,
  bgcolor=LightGray,
  linenos,
  ]{c}
int main() {
  int *x = malloc(2*8);
  x[0] = 'a';
  x[1] = 'b';
  putchar((*(x++))++);
}
\end{minted}\par
La structure de donnée utilisée pour stocker les variables présentes dans
l'environnement est impérative. Elle consiste d'une pile de tables de hashage,
correspondant aux portées imbriquées.

\section*{Génération d'erreurs}
Les erreurs syntaxiques sont fatales, dans le sens que le compilateur s'arrête
à la première, sans essayer de typer ce qui a été compris. Cependant, pour rendre
les messages d'erreur plus compréhensibles, le compilateur suggère toujours
comment corriger une erreur de syntaxe. La qualité de la suggestion dépend du
nombre de façons de corriger l'erreur: moins il y en a, plus le compilateur peut
être précis dans sa suggestion.\par
Les autres erreurs sont accumulées dans une variable statique. Juste avant la génération
du code (c'est-à-dire, au dernier moment où il n'y a plus d'erreurs), s'il y a
des erreurs, elles sont affichées, et le compilateur s'arrête.\par
Des efforts sont faits pour mettre en contexte les messages d'erreur, y compris
quand plusieurs parties du code amènent à une erreur (et pas uniquement la
dernière partie du code qui a amené à l'erreur), et l'afficher joliment.\par
Le sucre syntaxique interfère légèrement avec l'affichage des erreurs, dans la
mesure où il empêche d'afficher un message d'erreur spécifique à \(e[i]\) si
cette expression est mal typée.

\section*{Annotation de compilation}
Une fois le code typé, on peut l'annoter avec des informations utiles à la compilation.
Notamment, on transforme toutes les variables en couple (offset, profondeur), offset représentant
le décalage par rapport au tableau d'activation et profondeur la profondeur de la fonction parent dans
laquelle elle a été déclarée. De plus, les opérations arithmétiques sur les pointeurs sont modifiées afin
de prendre en compte la taille du type (\verb|p + 1| sera traduit en \verb|p + 8| si \verb|p| est
un pointeur, \verb|++p| incrémentera \verb|p| de \verb|8|, etc.). On associe également un identifiant
unique à chaque construction \verb|if|, \verb|while| et \verb|for|, pour pouvoir générer des labels
uniques pour les sauts. Pour les \verb|if|, c'est \verb|ifline.col| avec \verb|line| la ligne et
\verb|col| la colonne du début du \verb|if|, etc. C'est bien unique car un unique \verb|if|, etc peut
commencer à la même position. De plus ces labels permettent de cerner facilement d'où vient un problème
dans l'assembleur ou le code produit produit. Les expressions booléennes devant s'évaluer paresseusement,
il est également nécessaire de faire un saut dans l'asm produit : c'est pourquoi on génère le label
avec l'emplacement de la position de fin du premier argument (la position de début du second argument
aurait également convenu). En effet, pour une expression comme \verb|a && b && c| qui
est parsée comme \verb|(a && b) && c|, le \verb|&&| intérieur et le \verb|&&| extérieur partagent
la même position de début. En revanche, le premier caractère qui n'est pas une espace
ou un saut de ligne après le deuxième argument doit être l'opérateur (\verb|&&| ou \verb?|?),
et il y a donc au plus une expression qui sera associée à chaque position,
ce qui rend bien le label unique. Enfin,
on construit un arbre de dépendances des fonctions pour savoir dans quelle fonction a été déclarée
telle fonction et on calcule pour chaque bloc la taille maximum de pile dont il a besoin pour stocker
des variables pour qu'à la compilation, une fonction puisse allouer en une fois toute la pile
dont elle a besoin. Cela ne prend pas en compte la pile temporairement allouée pour les calculs.

\section*{Génération de code}
La génération du code repose sur la librairie \verb|write_x86_64| de Samuel.\par
La pile n'est pas alignée en général avant un appel de fonction, donc un petit
préambule est généré avant chaque appel à \verb|putchar| et \verb|malloc|.\par
Le résultat de l'évaluation d'une expression est stockée dans \verb|rax|, ce qui
en particulier évite de bouger le résultat des appels à \verb|malloc|. La pile
utilisée par chaque fonction est allouée en une seule fois. Pour accéder à l'addresse
d'une variable, un petit utilitaire \verb|push_addr| se charge de calculer à combien de parent
il faut remonter pour trouver le bon tableau d'activation. Enfin, pour un appel de fonction,
grâce à l'arbre de dépendances, on peut trouver la profondeur du tableau d'activation de la
fonction appelée grâce à une recherche de LCA (lowest common ancestor) dans l'arbre, sachant
que la fonction appelée est forcément un fils direct de ce LCA.

\end{document}
